\section{Background of the Study}

Waste management has been acknowledged since ancient times, as demonstrated by early cities like Rome and Athens. Nevertheless, this mainly focuses on removing the waste rather than recycling or segregation. With time, especially during industrialization and city growth processes, there was an increase in quantities of waste leading to more systematic waste management approaches. It was only at the end of the 20th century that waste categorization emphasizing biodegradable, non-biodegradable, and recyclable materials became important hence transitioning towards recycling and usability in modern waste management systems.

In today's world, waste management has become a critical global issue due to the rapid increase in urbanization, industrialization, and population growth. Especially in the Philippines, where it has been reported by \textcite{statista2022waste} to increase by more than 59.24 thousand tons per day in 2022, Poor waste segregation, contributes much to environmental degradation through soil and marine pollution. Various policies have been implemented to encourage proper waste disposal. However, public awareness and participation in proper segregation practices are still at a minimal rate.

Since mobile phone technology is increasingly becoming part of people's lives, it can provide a way to address such practical issues like waste management. This research proposes that a mobile application that will use image recognition technology for sorting waste into biodegradable, non-biodegradable, and recyclable categories should be developed to help improve waste segregation practices. Through technology usage, this research aims to encourage people to segregate their waste and create environmentally conscious behaviors properly.

Previous studies have explored various waste management approaches, such as manual classification, recycling programs, and automated sorting systems. However, research on mobile applications for real-time waste classification using image recognition is limited. For instance, a study by \textcite{Malik2022} demonstrated the potential of using Convolutional Neural Networks (CNNs) for classifying waste materials through images, showcasing the effectiveness of deep learning models in waste sorting. Despite these, a comprehensive waste classification system integrating real-time image recognition, particularly for mobile devices, is still lacking. Additionally, a study by \textcite{AhmadFaudzi2023}, demonstrates that effective user experience and interaction design in mobile applications could enhance user engagement and positively influence behavior change. Building on these insights, this study aims to address this existing gap by developing an educational and practical mobile application that incorporates real-time image recognition for daily waste segregation purposes.

%Cite your references. For instance,
%\begin{enumerate}
%\item In-line or text citations
%\begin{itemize}
%\item According to \textcite{BooktagYear}, the . . .
%\item \textcite{JournaltagYear} showed that the . . .
%\item As discussed by \textcite{InbookTagYear}, the . . .
%\end{itemize}

%\item Post-paragraph or parenthetical citations
%\begin{itemize}
%\item The fact is . . . blah blah blah \parencite{WebsiteTag}.
%\item Result result result . . . blah blah blah \parencite{ProceedingsTag}.
%\item Statement of fact . . . blah blah blah \parencite{ThesisTag}
%\end{itemize}
%\end{enumerate}
